\documentclass[12pt,a4paper]{leaflet}
\usepackage[utf8]{inputenc}
\usepackage[german]{babel}
\usepackage[T1]{fontenc}
\usepackage{microtype}
\usepackage{graphicx}
\usepackage{color}
\definecolor{gray}{gray}{.5}
\definecolor{darkgray}{gray}{.2}
\setlength{\parindent}{0pt} 
\usepackage{blindtext}
\usepackage[condensed]{tgheros}
\setmargins{2cm}{1cm}{8mm}{8mm}
\usepackage[absolute]{textpos}

\usepackage{hyperref}
\hypersetup{
    pdftoolbar=true,
    pdfmenubar=true,
    pdffitwindow=false,
    pdfstartview={FitH},
    pdfnewwindow=true,
    colorlinks=true,
    linkcolor=black,
    citecolor=black,
    filecolor=black,
    urlcolor=black
}
\def\UrlFont{\bfseries}
\pagestyle{empty} 
\AddToBackground*{2}{\put(0,25){\includegraphics[scale=1]{welle2}}}

%%% Softwaretemplate %%%
\newcommand{\software}[7]{
\begin{minipage}[t][0.4\textheight]{1\textwidth}
\flushleft
\section*{#1 \hfill \textcolor{gray}{#2}}
#3 
\medskip

%\scalebox{.9}[1.0]{\textit{Betriebssystem:} \textbf{#4}} \\
\textit{Betriebssystem:} \textbf{#4} \\
\textit{Lizenz:} \textbf{#6} \\
\textit{Deutschsprachig:} \textbf{#7} \\
\textit{Homepage:} \url{#5} \\
\vglue  2 true cm
\end{minipage}
}

% Hack for LibreOffice/OpenOffice: Version of \software without link
\newcommand{\softwaretwo}[7]{
\begin{minipage}[t][0.4\textheight]{1\textwidth}
\flushleft
\section*{#1 \hfill \textcolor{gray}{#2}}
#3 
\medskip

%\scalebox{.9}[1.0]{\textit{Betriebssystem:} \textbf{#4}} \\
\textit{Betriebssystem:} \textbf{#4} \\
\textit{Lizenz:} \textbf{#6} \\
\textit{Deutschsprachig:} \textbf{#7} \\
\textit{Homepage:} \textbf{#5} \\
\vglue  2 true cm
\end{minipage}
}


%%% Frontpage %%%
\newcommand{\frontpage}[1]{

\begin{titlepage}
\begin{flushleft}
\vglue  2 true cm

\textsc{\Huge Freie Software} 
\newcommand{\HRule}{\rule{0.9\linewidth}{0.5mm}}
\HRule \\[0.4cm]
\textsc{\textbf{\Huge #1}}

\vglue  2 true cm

{ \large\it
Acht Freie \\
Computerprogramme, \\
die Deine Rechte \\
als Nutzer/-in \\
respektieren \\
und schützen. \\
}
\end{flushleft}
\vfill
\begin{center}\includegraphics[scale=0.15]{cwfs}\end{center}
\begin{flushright}
\begin{minipage}{3.5cm}
\center
\includegraphics[scale=0.15]{plussy} \\
\bf FSFE-Fellowship-Gruppe\\ München
\end{minipage}
\end{flushright}
\end{titlepage}

\setlength{\TPHorizModule}{\textwidth}
\setlength{\TPVertModule}{\textheight}
\begin{textblock}{1}(0.1,1)
\small\textcolor{darkgray}{
Impressum: \\
FSFE Fellowship-Gruppe München \\
c/o Free Software Foundation Europe e.V. \\
Schönhauser Allee 6/7 \\
10119 Berlin}
\end{textblock}

\flushleft
}


%%% Backpage %%%
\newcommand{\backpage}{
\newpage
\section{Was ist »Freie Software«?}
»Freie Software« sind \textbf{Computer-Programme}, die unter einer Freien Software-Lizenz angeboten werden. Freie Software bietet dabei Möglichkeiten, die bei unfreier (proprietärer) Software meist zu Ungunsten der Nutzer/-innen eingeschränkt sind.

\textbf{Freie Software-Lizenzen} gewähren immer die folgenden wesentlichen \textbf{Freiheiten}:

\begin{itemize}
\item Die Freiheit, das Programm auf jede Weise und zu jedem beliebigen Zweck zu \textbf{verwenden}.
\item Die Freiheit, das Programm zu \textbf{verstehen} und den eigenen Bedürfnissen \textbf{anzupassen}.
\item Die Freiheit, das Programm \textbf{weiterzugeben}, um Deinen Mitmenschen zu helfen.
\item Die Freiheit, das Programm zu \textbf{verbessern} und diese Verbesserungen zu veröffentlichen.
\end{itemize}

Für einige dieser Freiheiten ist der Zugang zum \textbf{Quelltext} die Voraussetzung. 

Freie Software wird meist kostenfrei verbreitet, was aber kein Kriterium ist. Sie wird gegen Geld, Ruhm oder beides erstellt, jedoch niemals umsonst. :-)

Weitere Informationen zu Freier Software,
Freien Software-Lizenzen und dem »Free Software Movement«
findest Du unter \url{https://fsfe.org}.

}
